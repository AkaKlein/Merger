\documentclass{fmetfm}

%%%%%%%%%%packages
\usepackage[catalan,spanish,english]{babel}
%%%%%%%%%%%%%%%%%%%%%%%%%%%%%%%%%%%%%%%%%
%%%%%%%%%%%%%%%%%%%%%%%%%%%%%%%%%%%%%%%%
\usepackage[latin1]{inputenc}
%%%%%%%%%%%%%%%%%%%%%%%%%%%%%%%%%%%%%%%%
%%%%%%%%%%%%%%%%%%%%%%%%%%%%%%%%%%%%%%%%%
\PassOptionsToPackage{usenames,dvipsnames}{xcolor}
\usepackage[dvips]{graphicx,psfrag,overpic,color} %Necesario para figuras
\usepackage{latexsym} %Necesario para simbolos ampliados
\usepackage{amssymb} %Necesario para simbolos AMS
\usepackage{amsmath}%Necesario para simbolos AMS extra
\usepackage{amsfonts}%Necesario para fuentes AMS extra
\usepackage{amsthm}
\usepackage{pstricks}  % since the dash is rendered by pstricks!
\usepackage[postscript]{ucs}
\usepackage{fancyhdr}
\usepackage{listings}
\usepackage[]{algorithm2e}
\usepackage[a4paper]{geometry}
\usepackage{url}
\usepackage{pdfpages}
\usepackage{mathtools}

\newtheorem{theorem}{{Theorem}}
\newtheorem{prop}{{Proposition}}
\newtheorem{defi}{{Definition}}
\newtheorem{claim}{{Claim}}

\makeindex

\begin{document}


\selectlanguage{english}


\title{Library-Free Technology Mapping for VLSI Circuits with Regular Layouts}


\def\autor{Andrea Dana Klein Villalba}

\def\treball{Master's Degree Thesis}

\def\advisor{Jordi Cortadella Fortuny, Sachin Sapatnekar}

\vfill

\def\departament{Departament de Llenguatges i Sistemes Inform\`atics}



%-------------------------------------------------------------%
\maketitle



%-------------------------------------------------------------%
%%%Dedicatoria (opcional)
%-------------------------------------------------------------%
\dedicatoria{To my family, advisors and everyone that has made this thesis possible}

\dedicatoria{\textit{If you find that you're spending almost all your time on theory, start turning some attention to practical things; it will improve your theories. If you find that you're spending almost all your time on practice, start turning some attention to theoretical things; it will improve your practice.}}

\hfill Donald Knuth

%-------------------------------------------------------------%
%-------------------------------------------------------------%
%%%Prefaci (opcional)
%-------------------------------------------------------------%

%\begin{preface}
%\thispagestyle{empty}

%\end{preface}


%\clearpage

%---------------------------------------------------------------
% Abstract: un resum del contingut del treball (sobre 500 paraules)
% Ha de contenir una relacio de paraules clau i dels codis MSC
%(Math. Subject Classification 2000)
% Ha d'estar tambe en angles.
%---------------------------------------------------------------
\pagenumbering{roman}
\setcounter{page}{5}
\markboth{}{}

\begin{abstracteng}

\keywordseng{algorithms, technology mapping, graphs, optimization}
\msc2000{68W35, 68R10}

\vspace{1cm}
%%%Here the text

Technology mapping is the task to transform a technology independent logic network into a mapped network using gates from a library, optimizing some objective function such as total area, delay or power consumption. As stated, the problem is completely intractable. Therefore, different techniques are applied to solve the problem, such as using different simplified representations.

The usual approach consists of using a fixed library, typically partially handmade. Designing such libraries is a costly process, specially because of the lack of good automatic techniques to perform it. This is the main motivation for this work and the goal is to move from fixed-library technology mapping to library-free technology mapping. 

The idea of this is simple: design a new and specific library for each network instead of using a fixed library for all of them and then perform the technology mapping. Although simple, this idea presents several difficulties to overcome. As technology mapping is a hard problem, libraries have to be small enough to be tractable and the models used to generate them need to be both efficient and accurate enough. Finally, networks may have sizes of about billions nowadays, which means that the algorithms for all the process have to be very efficient.

This thesis presents different algorithms and techniques to move to library-free technology mapping for area and for delay optimization. The structure of this document is as follows. The first chapter contains an introduction to the problem and to the physical design of gates using transistors. Chapter 2 is focused on area optimization while chapter 3 covers delay optimization. Last chapter presents the results obtained using those methods and some conclusions.
 

\end{abstracteng}


\cleardoublepage

%--------------------------------------------------------------
% Taula de continguts
% S'elabora automaticament
%-------------------------------------------------------------
 \tableofcontents
 \cleardoublepage
%------------------------------------------------------------
% Cos del treball
%Es poden fer servir Parts, Capitols, Seccions i Subseccions
%-------------------------------------------------------------
\pagenumbering{arabic}

\chapter{Introduction}
This thesis is centered in the study of library-free technology mapping. The standard problem of technology mapping is the following: given a library of gates $L$ and a Boolean network $G$, represent $G$ as a network constructed using just the gates in $L$. This new network is called mapped network. But the technology mapping does not consist only in finding such a representation, but in minimizing some objective function (typically area, delay or power consumption).

Technology mapping usually uses a fixed library, which in most cases is either totally or partially handmade designed. The goal of this thesis is overcoming those two properties of libraries: instead of using fixed libraries, library-free technology mapping techniques are presented. The term library-free means that the library is generated specially for the circuit to be designed. Therefore, that obliges to develop automatic methods for constructing those libraries instead of designing them by hand.

The next sections include a brief introduction to the design of gates at physical level, as this is a key point to understand the rest of the work. The next two chapters will be focused on the minimization of area (Chapter 2) and delay (Chapter 3). Chapter 4 presents the results along with conclusions on future work.

\section{The design of gates using transistors}

In the project, circuits are going to be present at every moment, starting at the level of gates. Therefore, it is important to have a basic idea on how gates are designed and as the main element is the MOS transistor, a brief introduction to their types and how they work is presented here.

A MOSFET (Metal-Oxide-Semiconductor Field-Effect Transistor) is the most common transistor in circuits nowadays. A cross section of a general MOSFET is shown in Figure \ref{mosfet}. W.l.o.g. MOS transistors can be seen as switches.


Depending on how they work, there are two different types of MOS transistors: pMOS and nMOS. The difference is based on the type of doping. Pure semiconductors are called intrinsic. Doping refers to the process of intentionally adding impurities to the semiconductors to modify their electrical properties, obtaining extrinsic semiconductors. n-type semiconductors are those doped to have a larger electron concentration than hole concentration while p-type semiconductors have a larger hole concentration than electron concentration.

Figure \ref{cmos} shows a cross section of a pMOS and an nMOS transistor. In the case of an nMOS, both the source and the drain are n+ regions (the ``+'' sign is just a notation to indicate that the semiconductor is highly doped) and the body is a p region. Therefore, by applying a positive voltage to the gate, the electrons of the body are attracted, thus forming a channel from source to drain. pMOS transistors work just the opposite, as shown in the figure.

The transistor in the upper part is of type p and the other of type n. $Vdd$ is the  power supply voltage (logic 1) and $Vss$ stands for the logic 0 or ground (also noted as GND). Now let us analyze what happens with the value of $Z$ depending on the value of $A$. If $A = 1$, the n transistor acts as a closed switch, but the p transistor acts as an open one. Then, the logic 0 propagates to $Z$. If $A = 0$, then the state of the transistors is reversed, thus propagating a 1.

This scheme can be used to produce any logic function $f$ in which every input is negated and there are no negations affecting anything apart from single variables. De Morgan's laws are useful to put a function in that way. 

\begin{theorem}[De Morgan's laws] If $x$ and $y$ are boolean variables , then
\begin{eqnarray*}
(xy)' &=& x' + y'\\
(x + y)' &=& x'y'
\end{eqnarray*}
\end{theorem}

where the $'$ operator just denotes the negation of the Boolean formula affected.

Then, consider $f(x_1, \dots, x_n)$ such that $x_i$ appears always negated in the function for every $i$. $f$ can be recursively constructed using pMOS transistors, as Figure \ref{cmos_ops} shows:

\begin{itemize}
\item $f(x_1, \dots, x_n) = x_i'$. Then a single p transistor with input $x_i$ encodes that.
\item $f(x_1, \dots, x_n) = f_1(x_1, \dots, x_n)f_2(x_1, \dots, x_n)$. Then connect in series the circuits for $f_1$ and $f_2$.
\item $f(x_1, \dots, x_n) = f_1(x_1, \dots, x_n) + f_2(x_1, \dots, x_n)$. Then connect in parallel the circuits for $f_1$ and $f_2$.
\end{itemize}

That construction leads to a circuit which connects the logic 1 to the output when the function evaluates to one and acts as an open switch otherwise.

Considering $f'$ and by applying De Morgan's laws, a function with no negations is obtained and therefore, with every variable appearing with positive sign. With a similar procedure a circuit for $f'$ using nMOS transistors can be constructed: the rules for the addition and the product are equivalent, and the case of a single variable (which now is not negated) is designed with a single n transistor.

The circuit for $f$ is called the pull-up network and the circuit for $f'$ is called the pull-down network and one is the dual of the other. Notice that as they are complementary, for each possible combination of values for the inputs exactly one of them acts as an open switch and the other as a closed switch. The circuit obtained by connecting the outputs of $f$ and $f'$ realizes the function $f$. In Figure \ref{nand_nor} there are a couple of examples constructing a NAND and a NOR function of two inputs each.


With the series-parallel graph, it is easy to present the final form of the constructed gate and argue why the graph is important. To construct the gate the transistors need to be placed in some order, connected as the graph shows. Figure \ref{cmos_final} shows possible implementations for the graphs in Figure \ref{cmos_graph}, that can be seen as paths or cycles inside the graphs. For a complete design of a gate, see Figure \ref{NAND2_X1}. 

But as the reader may notice, such a path will not always exists. Therefore, sometimes is needed to add diffusion breaks to isolate a couple of chains of transistors. Figure \ref{diffusion_break} shows an example of such a case.

Therefore, the properties of the graph have a direct impact in the area of the physical implementation of the circuit. This will be covered in the section about area estimation in the next chapter.

This chapter has presented the process of technology mapping and how gates are physically designed. To finish this introduction, let us end with the design of a real circuit in terms of gates, i.e. a mapped network, as shown in Figure \ref{circuit}.




%\cleardoublepage \sloppy
%\begin{raggedright}
%\input{exemple_tfm.ind}
%\end{raggedright}



%--------------------------------------------------------------
%Seccions addicionals: Apendixs, Glosari, Bibliografia
%
%Els apendixs son com capitols pero van precedits de \apendix,
%es a dir: \apendix \chapter{Nom de l'apendix}
%
%\input{apendix1}

%Per a la bibliografia es pot fer servir els fitxers .bib
%(consulteu un manual de LaTex per al seu us)
%o simplement amb \begin{thebibliography}{100} \end{thebibliography}
%
%\bibliographystyle{amsplain}
%\bibliography{tfm}

%El glosari de termes es crea automaticament. Consulteu un manual
%de LaTex

%--------------------------------------------------------------

\begin{thebibliography}{9}

\bibitem{abc}
ABC: A System for Sequential Synthesis and Verification. Berkeley Logic Synthesis and Verification Group.
 
\bibitem{dc}
Design Compiler, from Synopsys. \url{http://www.synopsys.com/Tools/Implementation/RTLSynthesis/DesignCompiler/Pages/default.aspx}

\bibitem{cudd}
CUDD: CU Decision Diagram Package. Fabio Somenzi, Colorado University. \url{http://vlsi.colorado.edu/~fabio/CUDD/}.

\bibitem{elmore48}
W. C. Elmore. The Transient Response of Damped Linear Networks with Particular Regard to Wideband Amplifiers. Journal of Applied Physics, Vol. 19, 1948.

\bibitem{iscas85}
ISCAS85 Combinational Benchmark Circuits. \url{http://www.cbl.ncsu.edu/benchmarks/}.

\bibitem{rudellphd}
R. Rudell. Logic Synthesis for VLSI Designs. PhD thesis, University of California, Berkeley, 1989.

\bibitem{brayton89}
R. Brayton, R. Rudell, A. Sangiovanni-Vincentelli and A. Wang. Multilevel logic synthesis. Notes for lectures at Oxford/Berkeley Summer Engineering Programme, July 1989.

\bibitem{kk97}
Andreas Kuehlmann and Florian Krohm. Equivalence checking using
cuts and heaps. In DAC '97: Proceedings of the 34th annual conference
 on Design automation, pages 263?268, New York, NY, USA, 1997. ACM
  Press.

\bibitem{pl98}
Peichen Pan and Chih-Chang Lin. A new retiming-based technology
mapping algorithm for LUT-based FPGAs. In FPGA '98: Proceedings
 of the 1998 ACM/SIGDA sixth international symposium on Field pro-
grammable gate arrays, pages 35-42, New York, NY, USA, 1998. ACM
 Press.
 
\bibitem{bl70}
Burton H. Bloom. Space/time trade-offs in hash coding with allowable
errors. Commun. ACM, 13(7):422-426, 1970.

\bibitem{ck06}
Donald Chai and Andreas Kuehlmann. Building a better Boolean matcher and symmetry detector. DATE '06 Proceedings of the conference on Design, automation and test in Europe, pages 1079-1084. 

\bibitem{bula89}
O. Bula, J. Moser, J. Trinko, M. Weismann and F. Woytowich. Gross Delay Defect Evaluation for a CMOS Logic Design System Product. IBM Technical Memorandum, May 1989.

\bibitem{caisso91}
Jean-Paul Caisso, Eduard Cerny and Nicholas C. Rumin. A Recursive Technique for Computing Delays in Series-Parallel MOS Transistor Circuits. IEEE Transactions on Computer-Aided Design, Vol. 10, No. 5, May 1991, pages 589-595.

\bibitem{rockafellar}
R. Tyrrell Rockafellar. Convex Analysis. Princeton University Press. 1970.

\bibitem{fishburn}
Alfred E. Dunlop, John P. Fishburn, Dwight D. Hill and Donald D. Shugard. Experiments Using Automatic Physical Design Techniques for Optimizing Circuit Performance. 1990 IEEE.

\bibitem{nangate_library}
NanGate Open Cell Library. \url{http://www.nangate.com/?page_id=22}.
\end{thebibliography}

%
\end{document}
%--------------------------------------------------------------%
%--------------------------------------------------------------%
