\documentclass[12pt]{article}
\renewcommand{\baselinestretch}{1.5}

%%%%%%%%%%packages
\usepackage[english]{babel}
%%%%%%%%%%%%%%%%%%%%%%%%%%%%%%%%%%%%%%%%%
%%%%%%%%%%%%%%%%%%%%%%%%%%%%%%%%%%%%%%%%
\usepackage[latin1]{inputenc}
%%%%%%%%%%%%%%%%%%%%%%%%%%%%%%%%%%%%%%%%
%%%%%%%%%%%%%%%%%%%%%%%%%%%%%%%%%%%%%%%%%
\PassOptionsToPackage{usenames,dvipsnames}{xcolor}
\usepackage[dvips]{graphicx,psfrag,overpic,color} %Necesario para figuras
\usepackage{latexsym} %Necesario para simbolos ampliados
\usepackage{amssymb} %Necesario para simbolos AMS
\usepackage{amsmath}%Necesario para simbolos AMS extra
\usepackage{amsfonts}%Necesario para fuentes AMS extra
\usepackage{amsthm}
\usepackage{pstricks}  % since the dash is rendered by pstricks!
\usepackage[postscript]{ucs}
\usepackage{fancyhdr}
\usepackage{listings}
\usepackage[a4paper,left=3cm,right=3cm,top=2.5cm,bottom=2.5cm]{geometry}
\usepackage{url}
\usepackage{pdfpages}
\usepackage{mathtools}
\usepackage{titling}
\usepackage[nottoc,numbib]{tocbibind}
\usepackage{mathptmx}
\usepackage[explicit]{titlesec}


\titleformat{\section}[block]{\filcenter}{\thesection. \MakeUppercase{#1}}{1em}{}
\titleformat{name=\section, numberless}[block]{\filcenter}{\MakeUppercase{#1}}{1em}{}
\titleformat{\subsection}[hang]{}{}{1em}{\thesubsection. #1}
\titleformat{\subsubsection}[hang]{}{}{1em}{{\normalfont\thesubsubsection. \textit{#1}}}


\newtheorem{theorem}{{Theorem}}
\newtheorem{prop}{{Proposition}}
\newtheorem{defi}{{Definition}}
\newtheorem{claim}{{Claim}}

\makeindex

\pretitle{%
  \begin{center}
  \LARGE
  \vspace{1in}
  \includegraphics[width=4.56in]{logo_uab2}\\[\bigskipamount]
  \vspace{1.5in}
}
\posttitle{\end{center}}

\begin{document}


\selectlanguage{english}

\title{{\large TITLE: Merger Simulations: A Theoretical and Practical Approach\\
AUTHOR: Andrea Dana Klein Villalba\\
DEGREE: Economics\\
ADVISOR: Jordi Perdiguero Garc\'ia\\
DATE: June 7th, 2019\\}}
\date{}

%-------------------------------------------------------------%
\clearpage\maketitle
\thispagestyle{empty}
\newpage

\begin{abstract}
Hello, it's me,
\end{abstract}
\newpage

%--------------------------------------------------------------
% Taula de continguts
% S'elabora automaticament
%-------------------------------------------------------------
 \tableofcontents
 \newpage
%------------------------------------------------------------
% Cos del treball
%Es poden fer servir Parts, Capitols, Seccions i Subseccions
%-------------------------------------------------------------
\pagenumbering{arabic}

\section{Introduction}
I was wondering if after all these years you'd like to meet.

\subsection{Patata}
sdfsfs

\subsubsection{Frita}
sdfsdffs

\section{Linear Demands with Constant Costs}
Let $N$ be the number of products, and $\mathcal{N} = \{1, 2, \dots, N\}$.

Let $q_k (p)$ be the demand of the $k$-th good. Because it is a linear demand, its curve is defined as

\begin{equation*}
q_k(p) = a_k + \sum_{j=1}^N b_{kj}p_j
\end{equation*}

Let $m$ be a firm that produces $\mathcal{N}_m \subseteq \mathcal{N}$. Then, its profit is defined by

\begin{equation*}
\pi_m(p) = \sum_{j \in \mathcal{N}_m} (p_j - c_j)\cdot q_j(p)
\end{equation*}

The firm, then, seeks to maximize its profit by setting their prices according to the following formula

\begin{equation*}
\max_{\{p_j ~ | ~ j \in \mathcal{N}_m\}} \pi_m(p)
\end{equation*}

If we derive with respect to $p_k$ where $k\in \mathcal{N}_m$, then we obtain

\begin{eqnarray*}
\frac{\partial\pi_m(p)}{\partial p_k} &=& \frac{\partial}{\partial p_k} \sum_{j \in \mathcal{N}_m} (p_j - c_j) \cdot q_j(p) \\
&=& \sum_{j \in \mathcal{N}_m} \frac{\partial}{\partial p_k} \left[(p_j - c_j) \cdot q_j(p)\right] \\
&=& \sum_{j \in \mathcal{N}_m} \frac{\partial (p_j - c_j)}{\partial p_k} \cdot q_j(p) + \frac{\partial q_j(p)}{\partial p_k} \cdot (p_j - c_j)\\
&=& q_k(p) + \sum_{j \in \mathcal{N}_m} b_{jk} \cdot (p_j - c_j) 
\end{eqnarray*}

From here we obtain the first order condition

\begin{equation*}
q_k(p) + \sum_{j \in \mathcal{N}_m} b_{jk} \cdot (p_j - c_j) = 0
\end{equation*}

It is useful, now, to introduce a matrix $D$ of size $N \times N$ such that

\begin{equation*}
d_{ij} = \left\{
\begin{tabular}{ll}
1 & if $i,j$ produced by the same firm\\
0 & otherwise
\end{tabular}\right.
\end{equation*}

With $D$ we can write the previous first order condition as

\begin{equation*}
q_k(p) + \sum_{j = 1}^N d_{jk}b_{jk} \cdot (p_j - c_j) = 0
\end{equation*}

In this model each product is produced by exactly one firm. Therefore, we have one such first order condition for each product. Thus, we can write the system in a more compact way. 

\begin{equation*}
a + B^t p + (D \circ B)(p - c) = 0
\end{equation*}

where $\circ$ denotes the Hadamard product.
Isolating for $p$, we obtain the optimum prices

\begin{equation*}
p = (B^t + (D \circ B))^{-1}((D \circ B) c - a)
\end{equation*}


\section{Linear Demands with Linear Costs}

Let $N$ be the number of products, and $\mathcal{N} = \{1, 2, \dots, N\}$.

Let $q_k (p)$ be the demand of the $k$-th good. Because it is a linear demand, its curve is defined as

\begin{equation*}
q_k(p) = a_k + \sum_{j=1}^N b_{kj}p_j
\end{equation*}

Let $m$ be a firm that produces $\mathcal{N}_m \subseteq \mathcal{N}$. Then, its profit is defined by

\begin{equation*}
\pi_m(p) = \sum_{j \in \mathcal{N}_m} (p_j - c_j)\cdot q_j(p)
\end{equation*}

The firm, then, seeks to maximize its profit by setting their prices according to the following formula

\begin{equation*}
\max_{\{p_j ~ | ~ j \in \mathcal{N}_m\}} \pi_m(p)
\end{equation*}

Because costs are no longer constant, we must compute the marginal costs $c_j$. Then,

\begin{equation*}
c_j = \frac{\partial C(q_j)}{\partial q_j}
\end{equation*}

\begin{equation*}
C(q_j) = f + v \cdot q_j
\end{equation*}

Where $f$ is a constant fixed cost of production and $v$ is the variable cost arising from the production of an additional unit of $q_j$. 

\begin{figure}
\centering
\includegraphics[scale=0.5]{logo_uab}
\caption{Patata}
\end{figure}

\begin{table}[]
\begin{tabular}{lllll}
1 & 2  & 3 & 4 & 5 \\
  & 11 &   &   &   \\
  &    &   & 1 &   \\
  &    & 1 &   &  
\end{tabular}
\caption{Frita}
\end{table}

An example of reference is \cite{abc}.

\newpage
\begin{thebibliography}{9}

\bibitem{abc}
ABC: A System for Sequential Synthesis and Verification. Berkeley Logic Synthesis and Verification Group.
 
\bibitem{dc}
Design Compiler, from Synopsys. \url{http://www.synopsys.com/Tools/Implementation/RTLSynthesis/DesignCompiler/Pages/default.aspx}

\bibitem{cudd}
CUDD: CU Decision Diagram Package. Fabio Somenzi, Colorado University. \url{http://vlsi.colorado.edu/~fabio/CUDD/}.

\bibitem{elmore48}
W. C. Elmore. The Transient Response of Damped Linear Networks with Particular Regard to Wideband Amplifiers. Journal of Applied Physics, Vol. 19, 1948.

\bibitem{nangate_library}
NanGate Open Cell Library. \url{http://www.nangate.com/?page_id=22}.
\end{thebibliography}

\newpage
\listoffigures

\newpage
\listoftables

%
\end{document}
%--------------------------------------------------------------%
%--------------------------------------------------------------%
