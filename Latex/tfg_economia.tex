\documentclass[12pt]{article}
\renewcommand{\baselinestretch}{1.5}

%%%%%%%%%%packages
\usepackage[english]{babel}
%%%%%%%%%%%%%%%%%%%%%%%%%%%%%%%%%%%%%%%%%
%%%%%%%%%%%%%%%%%%%%%%%%%%%%%%%%%%%%%%%%
\usepackage[latin1]{inputenc}
%%%%%%%%%%%%%%%%%%%%%%%%%%%%%%%%%%%%%%%%
%%%%%%%%%%%%%%%%%%%%%%%%%%%%%%%%%%%%%%%%%
\PassOptionsToPackage{usenames,dvipsnames}{xcolor}
\usepackage[dvips]{graphicx,psfrag,overpic,color} %Necesario para figuras
\usepackage{latexsym} %Necesario para simbolos ampliados
\usepackage{amssymb} %Necesario para simbolos AMS
\usepackage{amsmath}%Necesario para simbolos AMS extra
\usepackage{amsfonts}%Necesario para fuentes AMS extra
\usepackage{amsthm}
\usepackage{pstricks}  % since the dash is rendered by pstricks!
\usepackage[postscript]{ucs}
\usepackage{fancyhdr}
\usepackage{listings}
\usepackage[a4paper,left=3cm,right=3cm,top=2.5cm,bottom=2.5cm]{geometry}
\usepackage{url}
\usepackage{pdfpages}
\usepackage{mathtools}
\usepackage{titling}
\usepackage[nottoc,numbib]{tocbibind}
\usepackage{mathptmx}
\usepackage[explicit]{titlesec}


\titleformat{\section}[block]{\filcenter}{\thesection. \MakeUppercase{#1}}{1em}{}
\titleformat{name=\section, numberless}[block]{\filcenter}{\MakeUppercase{#1}}{1em}{}
\titleformat{\subsection}[hang]{}{}{1em}{\thesubsection. #1}
\titleformat{\subsubsection}[hang]{}{}{1em}{{\normalfont\thesubsubsection. \textit{#1}}}


\newtheorem{theorem}{{Theorem}}
\newtheorem{prop}{{Proposition}}
\newtheorem{defi}{{Definition}}
\newtheorem{claim}{{Claim}}

\makeindex

\pretitle{%
  \begin{center}
  \LARGE
  \vspace{1in}
  \includegraphics[width=4.56in]{logo_uab2}\\[\bigskipamount]
  \vspace{1.5in}
}
\posttitle{\end{center}}

\begin{document}


\selectlanguage{english}

\title{{\large TITLE: Merger Simulations: A Theoretical and Practical Approach\\
AUTHOR: Andrea Dana Klein Villalba\\
DEGREE: Economics\\
ADVISOR: Jordi Perdiguero Garc\'ia\\
DATE: June 7th, 2019\\}}
\date{}

%-------------------------------------------------------------%
\clearpage\maketitle
\thispagestyle{empty}
\newpage

\begin{abstract}
Hello, it's me,
\end{abstract}
\newpage

%--------------------------------------------------------------
% Taula de continguts
% S'elabora automaticament
%-------------------------------------------------------------
 \tableofcontents
 \newpage
%------------------------------------------------------------
% Cos del treball
%Es poden fer servir Parts, Capitols, Seccions i Subseccions
%-------------------------------------------------------------
\pagenumbering{arabic}

\section{Introduction}

When we are first introduced into economic theory, we start talking about perfect competition models, where a single supplier's or consumer's decision does not affect the market. However, these models are far from the truth, which raises the question about how market power affects market equilibria.\\
This has been the conundrum that researchers and policy makers have faced and continue to do so. Theoretically, in the presence of market concentration, firms tend to perceive a higher level of profits, while consumer welfare is reduced. When we translate this theory into the real world, we find that most markets have some concentration of market power, and so we try to measure the impact that changes -or mergers- may have in it. Since this technique relies on quantitative estimations of demand and cost functions, there is some skepticism around it (\cite{peters} Peters, 2006). But even though it may be somewhat limited by the models, researchers do recognize its usefulness. Other researchers focus on the use of merger simulation and its implementation, particularly for antitrust and competition agencies, (\cite{ivaldi} \cite{bjorn} Ivaldi and Verboven, 2005; Bj\"{o}rnerstedt and Verboven, 2015), finding that the models do make correct predictions to an extent. \\
Another common issue is actually computing the mergers. When teaching oligopolies and mergers, we focus on markets where initially there are two producers, since that leads to relatively simple equations. These allow us to get an intuitive idea about what would happen in a market with three players, and so forth. However, when we want to analyze a market with more than two players, we find that our equations become considerably complex. This is where computer science comes into play, introducing programming into economic theory. This idea is far from new, and Davis (2006) \cite{davis} already defined a theoretical approach to compute mergers with linear demands and constant marginal costs.\\
Thus, in this project, we continue on Davis's (2006) \cite{davis} and we seek to create a tool that enables researchers, teaching staff, and students to easily compute and better understand the effects of a merger in a market with different characteristics. Therefore, we propose a program that does so for four different combinations of demands and costs, with pure algebraic methods and numerical methods when the former is not possible. We then move on to give user instructions, and move forward by proposing future applications and developments. 


\section{Models}

The models used in this project are characterized by Bertrand Competition, where the suppliers choose prices according to demand. This will be modeled as an economy in which there are $N$ products and $M$ different firms that produce them.

\begin{defi}
The set of products is denoted by $\mathcal{N} = \{1, 2, \dots, N\}$ and the subset of products produced by the same firm will be written as $\mathcal{N}_m \subseteq \mathcal{N}$. The set of firms will be written as $\mathcal{M}$.
\end{defi}

It is important to note that each product is exactly produced by one firm, i.e.  $\mathcal{N}_{m_1} \cap \mathcal{N}_{m_2} = \emptyset$ if $m_1$ and $m_2$ are two different firms and $\bigcup\limits_{m\in\mathcal{M}} \mathcal{N}_m = \mathcal{N}$. The firms in this economy will be profit-maximizing firms, and therefore we can write their profit function as
\begin{equation}
\pi_m(p) = \sum_{j \in \mathcal{N}_m} (p_j - c_j) \cdot q_j(p) \label{profit_eq}
\end{equation}

Where $p = (p_1, p_2, \dots, p_N)^t$ is the column vector of prices, and $c_j = \frac{\partial C_j}{\partial q_j}$ are the marginal costs and $q_j$ is the demand for the $j$-th good. 

\begin{defi}
The demand function $q$ is
\begin{eqnarray*}
q: ~ \mathbb{R}^N & \rightarrow & \mathbb{R}^N \\
p & \mapsto & q(p)
\end{eqnarray*}
where $q(p) = (q_1(p), q_2(p), \dots, q_N(p))^t$ is the column vector of the demands.
\end{defi}

\begin{defi}
The marginal cost function $c$ is
\begin{eqnarray*}
c: ~ \mathbb{R}^N & \rightarrow & \mathbb{R}^N \\
q & \mapsto & c(q)
\end{eqnarray*}
where $c(q) = (c_1(q_1), c_2(q_2), \dots, c_N(q_N))^t$ is the column vector of the marginal costs.
\end{defi}

Notice that, while $c(q)$ is a function of $q$, the latter is also a function of $p$, so $c(q)$ is a function of $p$ as well. 

For the given profit function, it is maximized when the first order conditions are fulfilled. Such that, deriving from equation \ref{profit_eq} with respect to $p_k$ such that $k \in \mathcal{N}_m$.
\begin{eqnarray}
\frac{\partial \pi_m}{\partial p_k} &=& \frac{\partial}{\partial p_k} \sum_{j \in \mathcal{N}_m} (p_j - c_j(q_j)) \cdot q_j(p) \\
&=& \sum_{j \in \mathcal{N}_m} \frac{\partial}{\partial p_k} \left[(p_j - c_j(q_j)) \cdot q_j(p)\right] \\
&=& \sum_{j \in \mathcal{N}_m} q_j(p) \frac{\partial (p_j - c_j)}{\partial p_k}(q_j) + \sum_{j \in \mathcal{N}_m} \frac{\partial q_j}{\partial p_k}(p) \cdot (p_j - c_j(q_j))\\
&=& q_k(p) - \sum_{j \in \mathcal{N}_m} q_j(p) \frac{\partial c_j}{\partial p_k} (q_j) + \sum_{j \in \mathcal{N}_m} \frac{\partial q_j}{\partial p_k}(p) \cdot (p_j - c_j(q_j)) = 0 \label{profit_FOC_eq}
\end{eqnarray}

Since each product is produced by exactly one firm, deriving as so for each product, we obtain a system of $N$ equations that is solved for the optimal prices $p^*$. In order to solve this system, it is useful to introduce the following variables:
\begin{defi}
Let $D = (d_{ij})$ be the ownership matrix of size $N \times N$, such that\\
\begin{center}
$d_{ij} = \left\{
\begin{tabular}{ll}
1 & if $i,j$ produced by the same firm\\
0 & otherwise
\end{tabular}\right.$
\end{center}
\end{defi}

By using the ownership matrix $D$, we can rewrite the first order condition from equation \ref{profit_FOC_eq} as follows
\begin{equation}
q_k(p) - \sum_{j=1}^N q_j(p) \frac{\partial c_j}{\partial p_k} (q_j)d_{jk} + \sum_{j=1}^N \frac{\partial q_j}{\partial p_k}(p) \cdot (p_j - c_j(q_j))d_{jk} = 0 \label{profit_FOC_with_Ds_eq}
\end{equation}

Finally, we need a measure of welfare. For the producer welfare (or surplus), we compute the profits for the prices $p^*$ and quantities $q^*$ at equilibrium. The consumer welfare $CW$, will be computed as follows
\begin{equation}
CS = \int_{p^*}^{p(q=0)} q(p) dp \label{consumer_welfare_eq}
\end{equation}

\subsection{Linear Demands with Constant Marginal Costs}

In this model we will assume that demands are linear, such that
\begin{equation*}
q_k(p) = a_k + \sum_{j=1}^N b_{kj}p_j
\end{equation*}

Because marginal costs are constant, the costs function $C_j$ will be characterized by
\begin{equation*}
C_j (q_j)= f_j + v_j \cdot q_j
\end{equation*}

Where $f_j$ is measure of fixed costs, $v_j$ is a measure of variable costs, and  $j \in \mathcal{N}$. Thus, marginal costs are
\begin{equation*}
c_j = \frac{\partial C_j}{\partial q_j}(q_j) = v_j
\end{equation*}

Computing the first order conditions from equation \ref{profit_FOC_with_Ds_eq}, we obtain
\begin{eqnarray*}
\frac{\partial\pi_m(p)}{\partial p_k}
&=& q_k(p) + \sum_{j = 1}^N d_{jk}b_{jk} \cdot (p_j - c_j) = 0
\end{eqnarray*}

Because we have one such first order condition for each product, we can write the system in a more compact way using matrices and vectors.
\begin{equation*}
a + B^t p + (D \circ B)(p - c) = 0
\end{equation*}

Where $\circ$ denotes the Hadamard product.
Isolating $p$, we obtain the optimum prices $p^*$
\begin{equation*}
p^* = (B^t + (D \circ B))^{-1}((D \circ B) c - a)
\end{equation*}

Then, optimum quantities are defined by
\begin{equation*}
q^* = a + B^t \cdot p^*
\end{equation*}

We are now interested in computing consumer welfare $CS$, following the form in equation \ref{consumer_welfare_eq}. 
\begin{equation*}
CS = \int_{p^*}^{-B^{t^{-1}} \cdot a} q(p) dq = \frac{(-B^{t^{-1}} \cdot a - p^*) \cdot q^*}{2} 
\end{equation*}

\subsection{Linear Demands with Linear Marginal Costs}

In this model we will assume that demands are linear, such that
\begin{equation*}
q_k(p) = a_k + \sum_{j=1}^N b_{kj}p_j
\end{equation*}

Because marginal costs are linear, the costs function $C_j$ will be characterized by
\begin{equation*}
C_j(q_j) = f_j + v_j \cdot q_j + w_j \cdot q_j^2
\end{equation*}

Where $f_j$ is a measure of fixed costs, $v_j$ is a measure of linear variable costs, $w_j$ is a measure of quadratic variable costs, and $j \in \mathcal{N}$ Thus, marginal costs are
\begin{equation*}
c_j = \frac{\partial C(q_j)}{\partial q_j}= v_j + w_j \cdot q_j.
\end{equation*}

Computing the first order conditions from equatio \ref{profit_FOC_with_Ds_eq}, we obtain
\begin{equation*}
\frac{\partial\pi_m(p)}{\partial p_k} = q_k(p) + \sum_{j = 1}^N b_{jk} \cdot d_{jk} \cdot p_j - \sum_{j = 1}^N b_{jk} \cdot d_{jk} \cdot v_j - 2 \cdot \sum_{j = 1}^N b_{jk} \cdot d_{jk} \cdot w_j \cdot q_j (p) = 0
\end{equation*}

Because we have one such first order condition for each product, we can write the system in a more compact way using matrices and vectors. 
\begin{equation*}
a + B ^ t p + (D \circ B)(p - v - 2 \cdot w \cdot (a + B^t p)) = 0
\end{equation*}

Isolating $p$, we obtain the optimum prices $p^*$
\begin{equation*}
p^* = [B^t + (D \circ B) - 2 (D \circ B)(w \circ B^t]^{-1} \cdot [-a + (D \circ B) v + 2 (D \circ B)(w \circ a)]
\end{equation*}

Then, the optimum quantites are defined by
\begin{equation*}
q^* = a + B^t \cdot p^*
\end{equation*}

We are now interested in computing consumer welfare $CS$, following the form in equation \ref{consumer_welfare_eq}.
\begin{equation*}
CS = 
\end{equation*}

\subsection{Quadratic Demands with Constant Marginal Costs}

In this model we will assume that demands are quadratic, such that
\begin{equation*}
q_k(p) = a_k + \sum_{j=1}^N b_{kj}p_j + \sum_{j=1}^N e_{kj} p_j^2
\end{equation*}


Because marginal costs are constant, $C_j$ will be characterized by 
\begin{equation*}
C_j(q_j) = f_j + vj \cdot q_j
\end{equation*}

Where $f_j$ is a measure of fixed costs, $v_j$ is a measure of variable costs, and $j \in \mathcal{N}$. Thus, marginal costs are,
\begin{equation*}
c_j = \frac{\partial C_j(q_j)}{\partial q_j} = v_j
\end{equation*}

Computing the first order conditions from equation \ref{profit_FOC_with_Ds_eq}, we obtain
\begin{equation*}
\frac{\partial \pi_m(p)}{\partial p_k} = q_j(p) + \sum_{j=1}^N (d_{jk} \cdot b_{jk} + 2 \cdot d_{jk} \cdot e_{jk} \cdot p_k) \cdot (p_j - c_j) = 0
\end{equation*}

Because we have one such first order condition for each product, we can write the system in a more compact way using matrices and vectors.
\begin{equation*}
a +(B \circ D) p + (E \circ D) (p \circ p) + ((B \circ D) + 2 (E \circ D) p) (p - c) = 0 
\end{equation*}

Because this equation cannot be isolated for $p$, we obtain the optimum prices through the Newton-Raphson Method.

\begin{equation*}
P_{n+1} = P_n - J^{-1} \cdot F(P_n)
\end{equation*}

Where $J^{-1}$ is the inverse of the Jacobian matrix, and $F_j(P_n) = \partial \pi_m / \partial p_k$. The Jacobian matrix is constructed as follows

\begin{equation}
J_{ij} = \frac{\partial F_i}{\partial p_j} = \frac{\partial q_i(p)}{\partial p_j} + \sum_{j \in \mathcal{N}_m} \frac{\partial^2 q_i(p)}{\partial p_i \partial p_j} + \sum_{j \in \mathcal{N}_m} \frac{\partial q_i(p)}{\partial p_j}
\end{equation}

Then, the optimum quantities are defined by 
\begin{equation*}
q^* = a + B^t p^* + E^t p^{*2}
\end{equation*}

We are now interested in computing consumer welfare $CS$ following the form in equation \ref{consumer_welfare_eq}.
\begin{equation*}
CS = 
\end{equation*}

\subsection{Quadratic Demands with Linear Marginal Costs}

In this model we will assume that demands are quadratic, such that
\begin{equation*}
q_k(p) = a_k + \sum_{j = 1}^N b_{kj} p_j + \sum_{j = 1}^N e_{kj}p_j^2
\end{equation*}

Because marginal costs are quadratic, $C_j$ will be characterized by
\begin{equation*}
C_j(q_j) = f_j + v_j \cdot q_j + w_j \cdot q_j^2
\end{equation*}

Where $f_j$ is a measure of fixed costs, $v_j$ is a measure of linear variable costs, $w_j$ is a measure of quadratic variable costs, and $j \in \mathcal{N}$. Thus, marginal costs are,
\begin{equation*}
c_j = \frac{\partial C_j(q_j)}{\partial q_j}= v_j + w_j \cdot q_j
\end{equation*}

Computing the first order conditions from equation \ref{profit_FOC_with_Ds_eq}, we obtain
\begin{equation*}
\frac{\partial \pi_m(p)}{\partial p_k} = q_k - \sum_{j = 1}^N (b_{jk} \cdot d_{jk} + 2e{jk} \cdot d{jk} \cdot p_j) w_j \cdot q_j + \sum_{j = 1}^N (p_j - v_j - w_j \cdot q_j)(b_{jk} \cdot d_{jk} + 2e_{jk} \cdot d_{jk} \cdot p_k = 0
\end{equation*}

Because we have one such first order condition for each product, we can write the system in a more compact way using matrices and vectors. 
\begin{eqnarray*}
a + B^t p + E^t (p \circ p) - ((B \circ D) &+& 2(E \circ D)p) (a + B^t p + E^t (p \circ p) w + \\
(p - v - w(a + B^t p &+& E^t (p \circ p)))((B \circ D) + 2 (E \circ D) p) = 0
\end{eqnarray*}

Because this equation cannot be isolated for $p$, we obtain the optimum prices through the Newton-Raphson Method.
\begin{equation*}
P_{n+1} = P_n - J^{-1} \cdot F(P_n)
\end{equation*}

Where $J^{-1}$ is the inverse of the Jacobian matrix, and $F_j(P_n) = \partial \pi_m / \partial p_k$. The Jacobian matrix is constructed as follows

\begin{eqnarray*}
J_{ij} = \frac{\partial F_i}{\partial p_j} = \frac{\partial q_i}{\partial p_j} &-& \sum_{k \in \mathcal{N}_m} \frac{\partial^2 q_k}{\partial p_i \partial p_j} w_k \cdot q_k \cdot d_{ik}\\
 - 2\sum_{k \in \mathcal{N}_m} w_k \cdot d_{ik} \frac{\partial q_k}{\partial p_i} \frac{\partial q_k}{\partial p_j} + \frac{\partial q_j}{\partial p_i} d_{ik} &+& \sum_{k \in \mathcal{N}_m} (p_k - v_k - w_k \cdot q_k) \frac{\partial^2 q_k}{\partial p_i \partial p_j}d_{ik}
\end{eqnarray*}

Then, the optimum quantities are defined by 
\begin{equation*}
q^* = a + B^t p^* + E^t p^{*2}
\end{equation*}

We are now interested in computing consumer welfare $CS$ following the form in equation \ref{consumer_welfare_eq}
\begin{equation*}
CS = 
\end{equation*}

\section{Conclusion}

An example of reference is \cite{abc}.

\newpage
\begin{thebibliography}{9}

\bibitem{abc}
ABC: A System for Sequential Synthesis and Verification. Berkeley Logic Synthesis and Verification Group.
 
\bibitem{dc}
Design Compiler, from Synopsys. \url{http://www.synopsys.com/Tools/Implementation/RTLSynthesis/DesignCompiler/Pages/default.aspx}

\bibitem{cudd}
CUDD: CU Decision Diagram Package. Fabio Somenzi, Colorado University. \url{http://vlsi.colorado.edu/~fabio/CUDD/}.

\bibitem{elmore48}
W. C. Elmore. The Transient Response of Damped Linear Networks with Particular Regard to Wideband Amplifiers. Journal of Applied Physics, Vol. 19, 1948.

\bibitem{nangate_library}
NanGate Open Cell Library. \url{http://www.nangate.com/?page_id=22}.

\bibitem{peters}
Peters, C. (2006). Evaluating the performance of merger simulation: Evidence from the US airline industry. The Journal of law and economics, 49(2), 627-649.

\bibitem{ivaldi}
Ivaldi, M., \& Verboven, F. (2005). Quantifying the effects from horizontal mergers in European competition policy. International Journal of Industrial Organization, 23(9-10), 669-691.

\bibitem{bjorn}
Bj\"{o}rnerstedt, J., \& Verboven, F. (2016). Does merger simulation work? Evidence from the Swedish analgesics market. American Economic Journal: Applied Economics, 8(3), 125-64.

\bibitem{davis}
Davis, P. (2006). Coordinated effects merger simulation with linear demands.
\end{thebibliography}


\newpage
\listoffigures

\newpage
\listoftables

%
\end{document}
%--------------------------------------------------------------%
%--------------------------------------------------------------%
