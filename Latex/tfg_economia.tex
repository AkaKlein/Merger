\documentclass{tfg}

%%%%%%%%%%packages
\usepackage[english]{babel}
%%%%%%%%%%%%%%%%%%%%%%%%%%%%%%%%%%%%%%%%%
%%%%%%%%%%%%%%%%%%%%%%%%%%%%%%%%%%%%%%%%
\usepackage[latin1]{inputenc}
%%%%%%%%%%%%%%%%%%%%%%%%%%%%%%%%%%%%%%%%
%%%%%%%%%%%%%%%%%%%%%%%%%%%%%%%%%%%%%%%%%
\PassOptionsToPackage{usenames,dvipsnames}{xcolor}
\usepackage[dvips]{graphicx,psfrag,overpic,color} %Necesario para figuras
\usepackage{latexsym} %Necesario para simbolos ampliados
\usepackage{amssymb} %Necesario para simbolos AMS
\usepackage{amsmath}%Necesario para simbolos AMS extra
\usepackage{amsfonts}%Necesario para fuentes AMS extra
\usepackage{amsthm}
\usepackage{pstricks}  % since the dash is rendered by pstricks!
\usepackage[postscript]{ucs}
\usepackage{fancyhdr}
\usepackage{listings}
\usepackage[a4paper]{geometry}
\usepackage{url}
\usepackage{pdfpages}
\usepackage{mathtools}

\newtheorem{theorem}{{Theorem}}
\newtheorem{prop}{{Proposition}}
\newtheorem{defi}{{Definition}}
\newtheorem{claim}{{Claim}}

\makeindex

\begin{document}


\selectlanguage{english}


\title{Merger Simulations: A Theoretical and Practical Approach}


\def\autor{Andrea Dana Klein Villalba}

\def\treball{Bachelor's Degree Thesis}

\def\advisor{Jordi Perdiguero Garc\'ia}

\vfill

\def\departament{Departament d'Economia Aplicada}



%-------------------------------------------------------------%
\maketitle



%-------------------------------------------------------------%
%%%Dedicatoria (opcional)
%-------------------------------------------------------------%
%\dedicatoria{To my family, advisors and everyone that has made this thesis possible}

%\dedicatoria{\textit{If you find that you're spending almost all your time on theory, start turning some attention to practical things; it will improve your theories. If you find that you're spending almost all your time on practice, start turning some attention to theoretical things; it will improve your practice.}}

%\hfill Donald Knuth

%-------------------------------------------------------------%
%-------------------------------------------------------------%
%%%Prefaci (opcional)
%-------------------------------------------------------------%

%\begin{preface}
%\thispagestyle{empty}

%\end{preface}


%\clearpage

%---------------------------------------------------------------
% Abstract: un resum del contingut del treball (sobre 500 paraules)
% Ha de contenir una relacio de paraules clau i dels codis MSC
%(Math. Subject Classification 2000)
% Ha d'estar tambe en angles.
%---------------------------------------------------------------
\pagenumbering{roman}
\setcounter{page}{5}
\markboth{}{}

\begin{abstracteng}

\keywordseng{}

\vspace{1cm}
%%%Here the text

Hello, it's me.
 

\end{abstracteng}


\cleardoublepage

%--------------------------------------------------------------
% Taula de continguts
% S'elabora automaticament
%-------------------------------------------------------------
 \tableofcontents
 \cleardoublepage
%------------------------------------------------------------
% Cos del treball
%Es poden fer servir Parts, Capitols, Seccions i Subseccions
%-------------------------------------------------------------
\pagenumbering{arabic}

\chapter{Introduction}
I was wondering if after all these years you'd like to meet.

\section{Linear Demands with Constant Costs}
Let $N$ be the number of products, and $\mathcal{N} = \{1, 2, \dots, N\}$.

Let $q_k (p)$ be the demand of the $k$-th good. Because it is a linear demand, its curve is defined as

\begin{equation*}
q_k(p) = a_k + \sum_{j=1}^N b_{kj}p_j
\end{equation*}

Let $m$ be a firm that produces $\mathcal{N}_m \subseteq \mathcal{N}$. Then, its profit is defined by

\begin{equation*}
\pi_m(p) = \sum_{j \in \mathcal{N}_m} (p_j - c_j)\cdot q_j(p)
\end{equation*}

The firm, then, seeks to maximize its profit by setting their prices according to the following formula

\begin{equation*}
\max_{\{p_j ~ | ~ j \in \mathcal{N}_m\}} \pi_m(p)
\end{equation*}

If we derive with respect to $p_k$ where $k\in \mathcal{N}_m$, then we obtain

\begin{eqnarray*}
\frac{\partial\pi_m(p)}{\partial p_k} &=& \frac{\partial}{\partial p_k} \sum_{j \in \mathcal{N}_m} (p_j - c_j) \cdot q_j(p) \\
&=& \sum_{j \in \mathcal{N}_m} \frac{\partial}{\partial p_k} \left[(p_j - c_j) \cdot q_j(p)\right] \\
&=& \sum_{j \in \mathcal{N}_m} \frac{\partial (p_j - c_j)}{\partial p_k} \cdot q_j(p) + \frac{\partial q_j(p)}{\partial p_k} \cdot (p_j - c_j)\\
&=& q_k(p) + \sum_{j \in \mathcal{N}_m} b_{jk} \cdot (p_j - c_j) 
\end{eqnarray*}

From here we obtain the first order condition

\begin{equation*}
q_k(p) + \sum_{j \in \mathcal{N}_m} b_{jk} \cdot (p_j - c_j) = 0
\end{equation*}

It is useful, now, to introduce a matrix $D$ of size $N \times N$ such that

\begin{equation*}
d_{ij} = \left\{
\begin{tabular}{ll}
1 & if $i,j$ produced by the same firm\\
0 & otherwise
\end{tabular}\right.
\end{equation*}

With $D$ we can write the previous first order condition as

\begin{equation*}
q_k(p) + \sum_{j = 1}^N d_{jk}b_{jk} \cdot (p_j - c_j) = 0
\end{equation*}

In this model each product is produced by exactly one firm. Therefore, we have one such first order condition for each product. Thus, we can write the system in a more compact way. 

\begin{equation*}
a + B^t p + (D \circ B)(p - c) = 0
\end{equation*}

where $\circ$ denotes the Hadamard product.
Isolating for $p$, we obtain the optimum prices

\begin{equation*}
p = (B^t + (D \circ B))^{-1}((D \circ B) c - a)
\end{equation*}


\section{Linear Demands with Linear Costs}

Let $N$ be the number of products, and $\mathcal{N} = \{1, 2, \dots, N\}$.

Let $q_k (p)$ be the demand of the $k$-th good. Because it is a linear demand, its curve is defined as

\begin{equation*}
q_k(p) = a_k + \sum_{j=1}^N b_{kj}p_j
\end{equation*}

Let $m$ be a firm that produces $\mathcal{N}_m \subseteq \mathcal{N}$. Then, its profit is defined by

\begin{equation*}
\pi_m(p) = \sum_{j \in \mathcal{N}_m} (p_j - c_j)\cdot q_j(p)
\end{equation*}

The firm, then, seeks to maximize its profit by setting their prices according to the following formula

\begin{equation*}
\max_{\{p_j ~ | ~ j \in \mathcal{N}_m\}} \pi_m(p)
\end{equation*}

Because costs are no longer constant, we must compute the marginal costs $c_j$. Then,

\begin{equation*}
c_j = \frac{\partial C(q_j)}{\partial q_j}
\end{equation*}

\begin{equation*}
C(q_j) = f + v \cdot q_j
\end{equation*}

Where $f$ is a constant fixed cost of production and $v$ is the variable cost arising from the production of an additional unit of $q_j$. 

%\cleardoublepage \sloppy
%\begin{raggedright}
%\input{exemple_tfm.ind}
%\end{raggedright}



%--------------------------------------------------------------
%Seccions addicionals: Apendixs, Glosari, Bibliografia
%
%Els apendixs son com capitols pero van precedits de \apendix,
%es a dir: \apendix \chapter{Nom de l'apendix}
%
%\input{apendix1}

%Per a la bibliografia es pot fer servir els fitxers .bib
%(consulteu un manual de LaTex per al seu us)
%o simplement amb \begin{thebibliography}{100} \end{thebibliography}
%
%\bibliographystyle{amsplain}
%\bibliography{tfm}

%El glosari de termes es crea automaticament. Consulteu un manual
%de LaTex

%--------------------------------------------------------------

\begin{thebibliography}{9}

\bibitem{abc}
ABC: A System for Sequential Synthesis and Verification. Berkeley Logic Synthesis and Verification Group.
 
\bibitem{dc}
Design Compiler, from Synopsys. \url{http://www.synopsys.com/Tools/Implementation/RTLSynthesis/DesignCompiler/Pages/default.aspx}

\bibitem{cudd}
CUDD: CU Decision Diagram Package. Fabio Somenzi, Colorado University. \url{http://vlsi.colorado.edu/~fabio/CUDD/}.

\bibitem{elmore48}
W. C. Elmore. The Transient Response of Damped Linear Networks with Particular Regard to Wideband Amplifiers. Journal of Applied Physics, Vol. 19, 1948.

\bibitem{iscas85}
ISCAS85 Combinational Benchmark Circuits. \url{http://www.cbl.ncsu.edu/benchmarks/}.

\bibitem{rudellphd}
R. Rudell. Logic Synthesis for VLSI Designs. PhD thesis, University of California, Berkeley, 1989.

\bibitem{brayton89}
R. Brayton, R. Rudell, A. Sangiovanni-Vincentelli and A. Wang. Multilevel logic synthesis. Notes for lectures at Oxford/Berkeley Summer Engineering Programme, July 1989.

\bibitem{kk97}
Andreas Kuehlmann and Florian Krohm. Equivalence checking using
cuts and heaps. In DAC '97: Proceedings of the 34th annual conference
 on Design automation, pages 263?268, New York, NY, USA, 1997. ACM
  Press.

\bibitem{pl98}
Peichen Pan and Chih-Chang Lin. A new retiming-based technology
mapping algorithm for LUT-based FPGAs. In FPGA '98: Proceedings
 of the 1998 ACM/SIGDA sixth international symposium on Field pro-
grammable gate arrays, pages 35-42, New York, NY, USA, 1998. ACM
 Press.
 
\bibitem{bl70}
Burton H. Bloom. Space/time trade-offs in hash coding with allowable
errors. Commun. ACM, 13(7):422-426, 1970.

\bibitem{ck06}
Donald Chai and Andreas Kuehlmann. Building a better Boolean matcher and symmetry detector. DATE '06 Proceedings of the conference on Design, automation and test in Europe, pages 1079-1084. 

\bibitem{bula89}
O. Bula, J. Moser, J. Trinko, M. Weismann and F. Woytowich. Gross Delay Defect Evaluation for a CMOS Logic Design System Product. IBM Technical Memorandum, May 1989.

\bibitem{caisso91}
Jean-Paul Caisso, Eduard Cerny and Nicholas C. Rumin. A Recursive Technique for Computing Delays in Series-Parallel MOS Transistor Circuits. IEEE Transactions on Computer-Aided Design, Vol. 10, No. 5, May 1991, pages 589-595.

\bibitem{rockafellar}
R. Tyrrell Rockafellar. Convex Analysis. Princeton University Press. 1970.

\bibitem{fishburn}
Alfred E. Dunlop, John P. Fishburn, Dwight D. Hill and Donald D. Shugard. Experiments Using Automatic Physical Design Techniques for Optimizing Circuit Performance. 1990 IEEE.

\bibitem{nangate_library}
NanGate Open Cell Library. \url{http://www.nangate.com/?page_id=22}.
\end{thebibliography}

%
\end{document}
%--------------------------------------------------------------%
%--------------------------------------------------------------%
